\documentclass{beamer}
\usetheme{Warsaw}

\usepackage{amsfonts} % math symobls
\usepackage{amsthm}
\usepackage[utf8]{inputenc}
\usepackage{polski}
\usepackage{graphics}
\usepackage{verbatim}
\usepackage{tikz}
\usetikzlibrary{arrows}
\usetikzlibrary{calc}

\newtheorem{tw}{Twierdzenie}
\newtheorem{lem}{Lemat}
\newtheorem{df}{Definicja}
\newtheorem{obs}{Obserwacja}  

\newcommand{\emp}[1]{\textcolor{blue}{\textit{#1}}}
\newcommand{\red}[1]{\textcolor{red}{#1}}
\newcommand{\tuple}[2]{\langle #1 , #2 \rangle}

\title{Fully Dynamic Strong Connectivity for Directed Graphs}
\subtitle{with Persistency}
\author{Michał Karpiński}
\date{8 stycznia, 2013}

\begin{document}

\begin{frame}[plain]
  \titlepage
\end{frame}

\begin{frame}{Silnie spójne składowe}

\begin{df}
{\bf Silnie spójna składowa} grafu skierowanego $G$ to taki maksymalny podgraf $H$, a jednocześnie jego spójna składowa, taka, że pomiędzy każdymi dwoma jej wierzchołkami istnieje ścieżka.
\end{df}

\begin{center}
\includegraphics[scale=0.5]{img/Scc.png}
\end{center}

\end{frame}

\begin{frame}{Specyfika problemu}

Problem: mając dany graf $G$, czy $u,v \in V(G)$ należą do tej samej silnie spójnej składowej?

\vspace{0.5cm}

Rozwiązania dla wersja statycznej są znane. Możemy dokonać preprocessing w czasie liniowym tak, aby zapytania wykonywały się w czasie stałym.

\end{frame}

\begin{frame}{Specyfika problemu (wersja dynamiczna)}

Różnica: graf $G$ zmienia się w czasie!

\vspace{0.5cm}

Cel: zbudowanie struktury danych wspierającej operacje:
\begin{itemize}
\item \emp{Update} - aktualizuje graf
\item \emp{Query} - odpowiada na pytanie o przynależność do komponentu
\end{itemize}
\end{frame}

\begin{frame}{Porządane procedury}

\begin{itemize}
\item \emp{Insert(E')} - tworzy nową \red{wersję grafu}, początkowo identyczną z poprzednią \red{wersją}, w której dodajemy zbiór krawędzi $E'$
\item \emp{Delete(E')} - usuwa zbiór krawędzi $E'$ ze \red{{\bf wszystkich} wersji grafu}
\item \emp{Query(u,v,i)} - sprawdza, czy $u$ i $v$ należą do wspólnego komponentu w $i$-tej wersji grafu
\end{itemize}

\end{frame}

\begin{frame}{Porządane procedury}
\begin{block}{Bardziej formalnie}
Algorytm zachowuje komponenty grafów $G_1,G_2 \cdots G_t$, gdzie $t$ jest liczbą operacji \emp{Insert} wykonaną do tej pory. Definiujemy $G_i=\tuple{V}{E_i}$ jako graf utworzony po $i$-tej operacji \emp{Insert}.
\end{block}

\begin{block}{Uproszczenie}
Zakładamy, że graf początkowy $G_0=\tuple{V}{E_0}$ jest grafem bez krawędzi, czyli $E_0 = \phi$. 
\end{block}
\end{frame}

\begin{frame}{Union-Find}
Prosta struktura do pamiętania zbiorów rozłącznych.
\begin{itemize}
\item \emp{Find(a)} - zwraca reprezentanta zbioru $a$. 
\item \emp{Union(Find(a),Find(b))} - łączy zbiory, zawierające $a$ i $b$
\end{itemize}
\begin{block}{Przypomnienie}
Łączny czas wykonania $m$ operacji \emp{Union} i \emp{Find} na zbiorach rozłącznych, zawierających w sumie $n$ elementów, jest równy $O(m\cdot\alpha(m,n))$, gdzie $\alpha$ jest odwrotnością funkcji Ackermanna.
\end{block}
\end{frame}

\begin{frame}{Lowest Common Ancestor}
\begin{obs}
Jeśli (w jakiś sposób) będziemy przechowywać las komponentów dla ciągu $G_0,G_1 \cdots G_t$, to operacja \emp{Query} może być zredukowana do zapytania \emp{LCA} na tym lesie.
\end{obs}

\begin{block}{Przypomnienie}
Wiadomo, że można wykonać preprocessing na lasie o liczbie wierzchołków \alert{$O(n)$}, w czasie \alert{$O(n)$} tak, aby zapytania \emp{LCA} wykonywać w czasie stałym \alert{$O(1)$}. Żródła:

\vspace{0.1cm}
{\small\textit{D. Harel and R. Tarjan. Fast algorithms for finding nearest common ancestros. SIAM Journal on Computing, 13:338-355, 1984.}}

\vspace{0.1cm}
{\small\textit{B. Shieber and U. Vishkin. On finding lowest common ancestors: Simplification and parallelization. SIAM Journal on Computing, 17:1253-1262, 1988.}}
\end{block}
\end{frame}

\begin{frame}{Dynamic edge partitioning}
\begin{df}
Mamy dane zbiory krawędzi $E_1 \subseteq E_2 \subseteq \cdots \subseteq E_t$. Dla każdego $i=1\dots t$, \textbf{dynamiczny zbiór krawędzi} $H_i$ grafu $G_i$ jest zdefiniowany jako:

\vspace{0.4cm}
{\small $H_i = \{(u,v)\in E_i | Query(u,v,i) \wedge (\neg Query(u,v,i-1) \vee (u,v) \notin E_{i-1})\}$}

\vspace{0.4cm}
oraz

\vspace{0.4cm}
{\small $H_{t+1} = E_t \setminus \cup^t_{i=1}H_i$}

\end{df}
\end{frame}

\begin{frame}{Dynamic set partitioning}
\begin{itemize}
\item $E_t = \cup^{t+1}_{i=1}H_i$
\item $H_i \cap H_j = \phi$ dla każdego $1 \leq i < j \leq t+1$
\item Zbiór $H_i$ jest złożony ze wszystkich krawędzi łączących dwa różne komponenty w $G_{i-1}$ lub takich, które nie znajdują się w $G_{i-1}$, ale łączą dwa różne komponenty w $G_i$.
\end{itemize}

\begin{block}{}
Taki podział jest \textbf{dynamiczny} ze względu na to, że każda zmiana w lesie komponentów może powodować przeniesienie krawędzi z $H_i$ do $H_j$, dla $i < j$.
\end{block}
\end{frame}

\begin{frame}{Algorytm}
Struktury:
\begin{itemize}
\item \emp{parent} - tablica przechowująca dla każdego wierzchołka w lesie, wskaźnik do jego ojca
\item \emp{version} - tablica przechowująca dla każdego wierzchołka w której wersji grafu po raz pierwszy pojawił się komponent związany z tym wierzchołkiem
\end{itemize}
\end{frame}

\begin{frame}{Algorytm - procedury}
\begin{center}
\includegraphics[scale=0.4]{img/Init.png}
\end{center}

\begin{obs}
Dwa wierzchołki $u$ i $v$ należą do tego samego komponentu w $G_i$ wtedy i tylko wtedy, gdy wersja ich najniższego wspólnego przodka jest mniejsza bądź równa $i$.
\end{obs}

\begin{center}
\includegraphics[scale=0.4]{img/Query.png}
\end{center}
\end{frame}

\begin{frame}{Algorytm - procedury}
\begin{center}
\includegraphics[scale=0.4]{img/Insert.png}
\end{center}

\begin{itemize}
\item \emp{FindScc($H_t$,$t$)} - aktualizuje las komponentów
\item \emp{Shift($H_t$,$H_{t+1}$)} - wydziela zbiór $H_{t+1}$ z $H_t$
\item \emp{Pre-LCA(parent)} - preprocesing dla zapytań LCA
\end{itemize}
\end{frame}

\begin{frame}{Algorytm - procedury}
\begin{center}
\includegraphics[scale=0.4]{img/FindScc.png}
\end{center}

\begin{center}
\includegraphics[scale=0.4]{img/Shift.png}
\end{center}
\end{frame}

\begin{frame}{Algorytm - procedury}
\begin{center}
\includegraphics[scale=0.4]{img/Delete.png}
\end{center}
\end{frame}

\begin{frame}{Algorytm - złożoność}
\begin{tw}
Każda operacja \emp{Insert} wykonuje się w pesymistycznym czasie $O(m \alpha(m,n))$, a każda operacja \emp{Delete} w zamortyzowanym czasie $O(m \alpha(m,n))$. Złożoność pamięciowa to $O(m + n)$.
\end{tw}
\end{frame}

\begin{frame}{Wypisanie rozkładu komponentów}

\begin{itemize}
\item rozpatrujemy dwa kolejne lasy (przed i po Delete)
\item dodajemy wskaźniki z każdego komponentu $v$ ze starego lasu do największysz komponentów w nowym lesie, które zawierają się w $v$
\end{itemize}

\end{frame}

\begin{frame}{Procedury}

\emp{Parts(v)} - zbiór największych komponentów nowego lasu, które zawierają się w $v$.

\vspace{0.5cm}

Innymi słowy: jeśli $v$ jest komponentem w starym lesie i $u \in Parts(v)$, to $u \subseteq v$. Dodatkowo jeśli istnieje komponent $u'$ nowego lasu taki, że $u \subseteq u'$, to $u' \not\subseteq v$.

\end{frame}

\begin{frame}{Procedury}
\begin{center}
\includegraphics[scale=0.4]{img/Parts.png}
\end{center}
\end{frame}

\begin{frame}{Procedury}
\begin{center}
\includegraphics[scale=0.4]{img/Split.png}
\end{center}
\end{frame}


\end{document}
